\documentclass[12pt,a4paper]{report}
\usepackage[italian]{babel}
\usepackage[utf8]{inputenc}
\usepackage{newlfont}
\usepackage{amsmath}
\usepackage{graphicx}
\usepackage{listings}
\graphicspath{ {./pic/} }

\begin{document}

\begin{titlepage}
  \begin{center}
    {{
      \Large{\textsc{Alma Mater Studiorum $\cdot$ Universit\`a di Bologna}}
    }} \rule[0.1cm]{15.8cm}{0.1mm}
    \rule[0.5cm]{15.8cm}{0.6mm}
    {\small{\bf SCUOLA DI SCIENZE\\
    Corso di Laurea in Informatica}}
  \end{center}
  \vspace{15mm}
  \begin{center}
    {\LARGE{\bf LA MIA FANTASTICA}}\\
    \vspace{3mm}
    {\LARGE{\bf OTTIMISTICA}}\\
    \vspace{3mm}
    {\LARGE{\bf TESI}}\\
  \end{center}
  \vspace{40mm}
  \par
  \noindent
  \begin{minipage}[t]{0.47\textwidth}
  {\large{\bf Relatore:\\
  Chiar.mo Prof.\\
  Claudio Sacerdoti Coen}}
  \end{minipage}
  \hfill
  \begin{minipage}[t]{0.47\textwidth}\raggedleft
  {\large{\bf Presentata da:\\
  Mattia Girolimetto}}
  \end{minipage}
  \vspace{20mm}
  \begin{center}
  {\large{\bf I Appello di Laurea\\%inserire il numero della sessione in cui ci si laurea
  Anno Accademico 2022-2023}}%inserire l'anno accademico a cui si è iscritti
  \end{center}
\end{titlepage}


\tableofcontents

\chapter{Introduzione}
\section{Dimostratori Interattivi di Teoremi, Matita e Dedukti}
Un dimostratore interattivo di teoremi (o \textit{proof assistant}) è un software 
che permette all'utente di costruire e verificare delle dimostrazioni matematiche formali.
Presa in input una prova espressa utilizzando uno specifico linguaggio 
formale, simile ad un linguaggio di programmazione, il software è in grado di verificarne
la correttezza.
In questo modo si possono costruire dimostrazioni in modo interattivo, controllando prograssivamente
la corretezza di ogni passo.
Uno dei benefici chiave dell'usare un dimostratore interattivo automatico è l'abilità di 
eliminare gli errori e le ambiguità che possono comparire nelle dimostrazioni tradizionali.




\end{document}
