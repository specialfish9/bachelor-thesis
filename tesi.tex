\documentclass[12pt,a4paper]{report}
\usepackage[italian]{babel}
\usepackage[utf8]{inputenc}
\usepackage{newlfont}
\usepackage{amsmath}
\usepackage{amsfonts}
\usepackage{graphicx}
\usepackage{listings}
\usepackage{styles/bussproofs}
\graphicspath{ {./pic/} }

\lstset{language=Caml}
\lstset{morekeywords={include,inductive} }
\lstset{basicstyle=\footnotesize\ttfamily,breaklines=true}

\begin{document}

\begin{titlepage}
  \begin{center}
    {{
      \Large{\textsc{Alma Mater Studiorum $\cdot$ Universit\`a di Bologna}}
    }} \rule[0.1cm]{15.8cm}{0.1mm}
    \rule[0.5cm]{15.8cm}{0.6mm}
    {\small{\bf SCUOLA DI SCIENZE\\
    Corso di Lira in Informatica}}
  \end{center}
  \vspace{15mm}
  \begin{center}
    {\LARGE{\bf DA MATITA A DEDUKTI}}\\
    \vspace{3mm}
    {\LARGE{\bf E RITORNO}}\\
  \end{center}
  \vspace{40mm}
  \par
  \noindent
  \begin{minipage}[t]{0.47\textwidth}
  {\large{\bf Relatore:\\
  Chiar.mo Prof.\\ % TODO 
  Claudio Sacerdoti Coen}}
  \end{minipage}
  \hfill
  \begin{minipage}[t]{0.47\textwidth}\raggedleft
  {\large{\bf Presentata da:\\
  Mattia Girolimetto}}
  \end{minipage}
  \vspace{20mm}
  \begin{center}
  {\large{\bf I Appello di Laurea\\%inserire il numero della sessione in cui ci si laurea
  Anno Accademico 2022-2023}}%inserire l'anno accademico a cui si è iscritti
  \end{center}
\end{titlepage}


\begin{abstract}
\end{abstract}

\tableofcontents

\chapter*{Introduzione}
I \textit{proof assistant} sono software in grado di verificare una dimostrazione
formale scritta da un utente. Questa viene rappresentata internamente attraverso 
termini di un linguaggio funzionale dotato di un sistema di tipi avanzato, tale
da renderlo equivalente a una logica. Il loro sviluppo negli anni ha portato ad una frammentazione
in quanto sistemi diversi, che usano logiche diverse, non sono compatibili tra 
loro. Per ovviare a questo problema un gruppo di ricercatori del INRIA ha creato
\textit{Dedukti}, la cui idea è quella di agire da \textit{type checker} per definizioni, enunciati e prove codificati
nel calcolo da lui implementato. Tale calcolo è sufficientemente espressivo 
per rappresentare in Dedukti qualunque sistema di tipi, ovvero qualunque linguaggio
funzionale con un sistema di tipi avanzato. Questo rende Dedukti un \textit{logical 
framework}, ovvero un calcolo che svolge il ruolo di meta-calcolo per la rappresentazione
uniforme di calcoli. Grazie ad esso è possibile sviluppare un sistema di codifica che
converta dimostrazioni scritte nel linguaggio di altri proof assistant, come \textit{Coq} e \textit{HOL lite},
nel metalinguaggio di Dedukti. Proprio per questo all'INRIA è stato sviluppato \textit{Krajono}, un
fork di \textit{Matita}, un proof assistant in sviluppo nel Dipartimento 
di Informatica dell'Università di Bologna. Krajono permette di esportare le definizioni
e i teoremi dati in Matita verso Dedukti, attraverso una codifica del calcolo
di Matita in quello di Dedukti. Krajono tuttavia non è più mantenuto e pertanto non integra
le funzionalità del Matita baseline sviluppate negli ultimi anni. 
Il lavoro di questa tesi è diviso in due parti: la prima prevede l'integrazione della 
funzionalità di esportazione di Krajono nel Matita odierno, mentre l'obiettivo della seconda
consiste nel implementare la possibilità di re-importare
codice Dedukti esportato precedentemente da Matita. Il secondo punto in particolare 
ha richiesto la risoluzione di una serie di problemi, in primis dovuti al fatto che
il linguaggio di Matita è più rigido e strutturato del metalinguaggio di Dedukti,
pertanto durante l'esportazione venivano perse informazioni necessarie alla
ricostruzione del codice originale. Ciò è stato risolto dotando l'export di
un meccanismo per preservarle dove necessario. In secondo luogo i sistemi di tipi
codificati in Dedukti, per essere consistenti, richiedono che le regole di riscrittura
scelte dagli sviluppatori per implementare l'encoding siano confluenti e normalizzanti.
Dedukti non effettua questa verifica, che demanda ad altri tool. Matita, invece, impone
un insieme molto restrittivo di controlli sintattici per garantire confluenza e terminazione,
che sono in generale proprietà indecidibili. Pertanto, quando si inverte il processo di
esportazione, sorge la necessità di ricostruire termini che soddisfino le condizioni 
sintattiche di Matita, che sono sufficienti, ma non necessarie, e che potrebbero essere
state violate durante l'encoding in Dedukti, nonostante le regole fossero anche confluenti
e normalizzanti.

\chapter{Basi teoriche}
In questo capitolo vengono affrontate le conoscenze teoriche necessarie per la
comprensione del lavoro esposto nei capitoli seguenti. Nella sezione \ref{teoriaDeiTipi}
vengono illustrati i fondamenti della \textit{Teoria dei Tipi}, teoria alla base 
di questa tesi. Nella sezione \ref{proofAssistant} vengono trattati i \textit{proof 
assistant} e il problema della loro interoperabilità.

\section{La Teoria dei Tipi} \label{teoriaDeiTipi}
\subsection{La Teoria dei Tipi}
La \textit{teoria dei tipi} è una branca della logica e dell'informatica teorica il
cui obbiettivo è quello di studiare i così detti \textit{type system}, ovvero
insiemi di regole che associano una proprietà chiamata \textit{tipo} a degli oggetti
chiamati \textit{termini}. Intuitivamente, assegnare un tipo ad un termine significa
assegnare al termine un'etichetta che rappresenta la natura del termine stesso.
Esempi comuni possono essere: 
\begin{itemize}
  \item $42$ è un numero naturale 
  \item $-5$ è un numero intero
  \item $falso$ è un valore di verità 
\end{itemize}

Formalmente si usa rappresentare queste espressioni separando il termine dal tipo usando
il simbolo '$:$'. 

Nella teoria dei tipi, anche le funzioni sono termini e possono essere
a loro volta tipizzate. Ad esempio, la seguente funzione rappresentata con un $\lambda$-termine
appartenente al $\lambda$-calcolo di Church 
\begin{center}
  $(\lambda$ $x$ $:$ $\mathbb{N}$ $.$ $x + x)$
\end{center}

è definita da $\mathbb{N}$ a $\mathbb{N}$, e pertanto ha tipo $\mathbb{N} \rightarrow \mathbb{N}$

\subsection{L'isomorfismo di Curry-Howard}
Durante il '900 i logici Haskell Curry e William Alvin Howard, scoprirono una
corrispondenza diretta tra prove formali e programmi. In particolare notarono
che gli operatori logici e le regole usate durante una dimostrazione formale
sono equivalenti a tipi e costrutti usati nei programmi scritti usando linguaggi
di programmazione funzionali. Ne segue che il verificare la correttezza di una
prova è corrisponde al verificare la correttezza degli assegnamenti di tipo di un
programma. Nella sua formulazione più generale, l'isomorfismo di Curry-Howard
può essere riassunto con la seguente tabella:

\begin{center}
  \begin{tabular}{ | c | c |}
    \hline
    \textbf{Logica} & \textbf{Informatica} \\
    \hline
    $\top$ & Tipo unit \\
    \hline
    $\bot $ & Tipo vuoto/void \\
    \hline
   $\wedge$ & Tipi prodotto \\  
    \hline
   $\vee$ & Tipi somma \\
    \hline
   $\Rightarrow$ & Tipi funzione \\ 
    \hline
   $\exists$ & Tipi $\Sigma$ \\ 
    \hline
    $\forall $ & Tipi $\Pi$ \\ 
    \hline

  \end{tabular}
\end{center}

\subsection{La teoria dei tipi nella pratica}
La teoria dei tipi trova quindi grande applicazione nel campo dell'informatica
grazie allo studio e allo sviluppo dei linguaggi di programmazione. Inoltre,
grazie all'isomorfismo di Curry-Howard, ha permesso lo sviluppo di dimostratori
interattivi di teoremi, i quali sono soggetto di questa tesi.

\section{Dedukti e Matita}\label{proofAssistant}
\subsection{Dimostratori Interattivi di Teoremi} 
Un dimostratore interattivo di teoremi (o \textit{proof assistant}) è un software 
che permette all'utente di costruire e verificare delle dimostrazioni matematiche
formali. Presa in input una prova espressa utilizzando uno specifico linguaggio 
formale, simile ad un linguaggio di programmazione, il software è in grado di
verificarne la correttezza. In questo modo si possono costruire dimostrazioni
in modo interattivo, controllando progressivamente la correttezza di ogni passo.
Uno dei benefici chiave dell'usare un dimostratore interattivo automatico è l'
abilità di eliminare gli errori e le ambiguità che possono comparire nelle 
dimostrazioni tradizionali. Durante la fase di verifica il proof assistant
rappresenta i termini della dimostrazione attraverso un linguaggio di programmazione
dotato di un sistema di tipi avanzato, ottenendo alla fine l'equivalente di un
programma. Successivamente esegue un processo di \textit{type checking} in cui
verifica la correttezza degli assegnamenti di tipo del programma. Grazie all'isomorfismo
di Curry-Howard questo equivale a verificare la correttezza della prova.

Il numero di proof assistant è aumentato nel tempo. Ciò porta sicuramente un
beneficio alla comunità scientifica, in quanto dimostra un crescente interesse
verso lo sviluppo di questi strumenti. Tuttavia questo aumento, unito al fatto
che proof assistant diversi sfruttano logiche diverse, porta ad una \textit{
frammentazione}. Per un utente non è quasi mai possibile infatti dimostrare la
veridicità di un teorema usando un proof assistant e usare la stessa dimostrazione
in un altro di questi tool senza doverla riscrivere.
Questo perché diversi proof assistant usano diversi sistemi di tipi e una dimostrazione
per poter essere verificata con sistema di tipi diverso deve essere tradotta.
Nasce dunque l'esigenza di favorire l'interoperabilità tra questi sistemi, in
modo da arginare questo problema e favorire lo sviluppo scientifico. 

\subsection{Dedukti}
Dedukti\footnote{"dedurre" in esperanto} è un \textit{logical framework}, ovvero
un meta-sistema che consente di definire logiche e teoremi usando tali logiche.
È sviluppato da alcuni ricercatori del \textit{Institut national de recherche
en informatique et en automatique} francese. Il software è open source, scritto
nel linguaggio di programmazione OCaml e distribuito secondo i termini della
CeCILL-B License. Uno degli obbiettivi principali di Dedukti quindi è creare
una connessione tra i diversi sistemi di dimostrazione assistita. Ciò significa
che le dimostrazioni possono essere tradotte da un sistema all'altro, agevolando
lo scambio e il riutilizzo delle dimostrazioni tra ambienti di lavoro diversi.
Alcuni proof system, come ad esempio Coq e Hol Lite \footnote{Coq: https://github.com/Deducteam/CoqInE,\\ HOL Lite: https://arxiv.org/pdf/1507.08720.pdf}
infatti godono già della possibilità di esportare e importare codice da e verso
Dedukti.

\paragraph{$\lambda\Pi$-Calcolo modulo}
Alla base di questo logical framework c'è il $\lambda\Pi$-calcolo modulo (o
semplicemente $\lambda\Pi$ modulo), un'estensione del $\lambda$-calcolo che introduce la \textit{tipizzazione dipendente} e le \textit{regole
di riscrittura}. La prima consente la specifica di tipi complessi che dipendono
dai valori delle espressioni, mentre le seconde consentono la trasformazione di
un'espressione in un'altra seguendo determinate sostituzioni o manipolazioni.
La sintassi del $\lambda\Pi$ modulo è la seguente
\begin{center}
  \textit{Termini } \hspace{1pt} A, B, t, u \hspace{1pt} $::=$ Kind | Type | $\Pi$ x : A . B | $\lambda$ x : A . B | A B | x \\
  \textit{Contesto} \hspace{1pt} $\Gamma$ \hspace{1pt} $::=$ $\emptyset$ | $\Gamma$, x : A | $\Gamma$, t $\hookrightarrow$ u
\end{center}

Mentre il sistema di tipi è definito dalle seguenti regole:

\begin{prooftree}
\AxiomC{}
\RightLabel{\textbf{Contesto vuoto}}
\UnaryInfC{[ ] \textit{ben formato}}
\end{prooftree}

\begin{prooftree}
\AxiomC{$\Gamma \vdash$ A : Type}
  \RightLabel{\textbf{Dichiarazione (Type)} $x \notin \Gamma$ }
\UnaryInfC{$\Gamma, x:A$ \textit{ben formato}}
\end{prooftree}

\begin{prooftree}
\AxiomC{$\Gamma \vdash$ A : Kind}
  \RightLabel{\textbf{Dichiarazione (Kind)} $x \notin \Gamma$ }
\UnaryInfC{$\Gamma,x:A$ \textit{ben formato}}
\end{prooftree}

\begin{prooftree}
\AxiomC{$\Gamma \vdash$ t $\hookrightarrow$ u \textit{ben formato}}
\RightLabel{\textbf{Regola}}
  \UnaryInfC{$\Gamma$, t $\hookrightarrow$ u \textit{ben formato}}
\end{prooftree}

\begin{prooftree}
\AxiomC{$\Gamma$ \textit{ben formato}}
\RightLabel{\textbf{Sorta}}
\UnaryInfC{$\Gamma \vdash$ Type : Kind}
\end{prooftree}

\begin{prooftree}
\AxiomC{$\Gamma$ \textit{ben formato}}
\AxiomC{x : A $\in \Gamma$}
\RightLabel{\textbf{Variabile}}
\BinaryInfC{$\Gamma \vdash$ x : A}
\end{prooftree}

\begin{prooftree}
\AxiomC{$\Gamma \vdash$ A : Type}
\AxiomC{$\Gamma,$ x : A $\vdash$ B : Type}
  \RightLabel{\textbf{Prodotto (Type)}}
\BinaryInfC{$\Gamma \vdash \Pi$ x : A . B  : Type}
\end{prooftree}

\begin{prooftree}
\AxiomC{$\Gamma \vdash$ A : Type}
\AxiomC{$\Gamma,$ x : A $\vdash$ B : Kind}
  \RightLabel{\textbf{Prodotto (Kind)}}
\BinaryInfC{$\Gamma \vdash \Pi$ x : A . B  : Kind}
\end{prooftree}

\begin{prooftree}
\AxiomC{$\Gamma \vdash$ A : Type}
\AxiomC{$\Gamma,$ x : A $\vdash$ B : Type}
\AxiomC{$\Gamma,$ x : A $\vdash$ t : B}
\RightLabel{\textbf{$\lambda$ astrazione}}
\TrinaryInfC{$\Gamma \vdash \lambda$ x : A . t  : $\Pi$ x : A . B}
\end{prooftree}

\begin{prooftree}
\AxiomC{$\Gamma \vdash$ t : $\Pi$ x : A . B}
\AxiomC{$\Gamma \vdash$ u : A}
\RightLabel{\textbf{Applicazione}}
  \BinaryInfC{$\Gamma \vdash$ (t u) : B $[$u/x$]$}
\end{prooftree}


\begin{prooftree}
\AxiomC{$\Gamma \vdash$ A : s}
\AxiomC{$\Gamma \vdash$ x : A}
\AxiomC{A $\equiv_{\beta\Gamma}$ B}
\RightLabel{\textbf{Equivalenza}}
\TrinaryInfC{$\Gamma \vdash$ x : B}
\end{prooftree}


\subsection{Matita}
Matita è un  proof assistant in sviluppo nel dipartimento di informatica dell'
Università di Bologna. È open source, scritto nel linguaggio di programmazione 
OCaml ed è rilasciato secondo i termini della GNU General Public Licence.
È basato sul \textit{calcolo delle costruzioni (co)induttive} (CIC), una teoria di tipi
dipendenti che estende il \textit{calcolo delle costruzioni} sviluppato da
Thierry Coquand aggiungendo i tipi induttivi, ovvero tipi auto-referenzianti.
Le regole di tipaggio del CIC per i tipi semplici sono quelle comuni dei linguaggi
di programmazione. Di seguito sono riportate quelle per le astrazioni e le applicazioni 

\begin{prooftree}
  \AxiomC{$\Gamma$, x : A $\vdash$ M : T}
  \RightLabel{\textbf{$\lambda$-astrazione}}
  \UnaryInfC{$\Gamma \vdash (\lambda $x : A . M) : A $\rightarrow$ B}
\end{prooftree}

\begin{prooftree}
  \AxiomC{$\Gamma \vdash$ f : A $\rightarrow$ B}
  \AxiomC{$\Gamma \vdash$ x : A}
  \RightLabel{\textbf{Applicazione}}
  \BinaryInfC{$\Gamma \vdash (f$ $x)$ : B}
\end{prooftree}


CIC è più ricco del $\lambda\Pi$ modulo usato da Dedukti in quanto gode di 
\textit{tipi induttivi}, \textit{punti fissi} e \textit{pattern matching}.
I tipi induttivi consentono la creazione di tipi complessi e autoreferenziali.
Possono essere utilizzati per definire strutture matematiche illimitate come liste,
alberi o grafi. L'esempio seguente mostra come si può definire una lista concatenata
di numeri naturali usando Matita. Una lista di questo tipo può essere vista come
una lista vuota oppure una coppia formata da un numero naturale e un'altra lista 
concatenata:

\begin{lstlisting}
include "basics/pts.ma".

inductive list : Type[0] =
   E : list
 | C : nat → list → list.

\end{lstlisting}

Invece punti fissi e il pattern matching sono utili per definire funzioni ricorsive
strutturali. Il primo ci permette di definire funzioni ricorsive, mentre il secondo
ci permette di ragionare per casi sulla struttura di un termine. Ad esempio, se
volessimo calcolare la lunghezza di una \texttt{list} dell'esempio precedente
si potrebbero fare delle considerazioni sulla sua struttura del tipo: \textit{qual è
la lunghezza della lista vuota?} \textit{Qual è la lunghezza di una lista formata 
da un numero naturale e un'altra lista?}. Il seguente codice Matita mostra una possibile
soluzione al problema di calcolare la lunghezza di una lista:

\begin{lstlisting}
let rec len l on l =
 match l with
 [ E ⇒ 0
 | C hd tl ⇒ 1 + len tl
 ].
\end{lstlisting}

Krajono permette di esportare e definizioni
e i teoremi dati in Matita verso Dedukti, attraverso una codifica del calcolo
di Matita in quello di Dedukti. Krajono tuttavia non è più mantenuto e pertanto non integra
le funzionalità del Matita baseline sviluppate negli ultimi anni.

\chapter{Da Matita a Dedukti} \label{capitoloExport}
In questo capitolo viene prima introdotto \textit{Krajono}, il fork di Matita
da cui proviene il codice responsabile dell'export da Matita verso Dedukti. 
Successivamente viene analizzato nel dettaglio tecnico e teorico, il funzionamento 
di questa esportazione. Infine ne vengono discusse problematiche e limitazioni.

\section{Krajono}
Un team di ricercatori del \textit{Institut national de recherche en informatique 
et en automatique} ha sviluppato Krajono \footnote{Significa "Matita" in Esperanto}
\footnote{https://github.com/Deducteam/Krajono}. Krajono ha la possibilità
di esportare definizioni e teoremi dati in Matita verso Dedukti. Per far ciò
sfrutta una codifica del calcolo di Matita nel metalinguaggio di Dedukti. 
Tuttavia attualmente questo fork non è più mantenuto da anni. Il primo passo
del mio lavoro è quello di integrare la funzionalità di export di Krajono
nel Matita baseline.

\subsection{Integrazione in Matita}
Per integrare le funzionalità di Krajono in Matita invece di aprire una \textit{
pull request} da una repository Git all'altra, si è preferito copiare e adattare i
singoli file sorgente responsabili dell'esportazione. Questo in quanto utilizzare
lo strumento di versionamento automatico avrebbe portato delle complicazioni in
quanto Krajono non è basato direttamente sul Matita baseline, ma su un ulteriore
fork, chiamato \textit{Matita with embedded Elpi}\footnote{https://github.com/LPCIC/matita}. 

In codice sorgente responsabile dell'esportazione è diviso in pochi moduli OCaml
quasi del tutto indipendenti dal restante codice sorgente di Matita, per tanto
integrarli nel Matita moderno non causa problemi.

\section{L'export}
\subsection{Il processo di esportazione}
\textit{Matitac} è il compilatore da linea di comando di Matita. Se lo si lancia
compila tutti i file con estensione \textit{.ma} presenti nella directory corrente. 
Opzionalmente, passando come argomento il nome di un file, è possibile anche 
compilare solo quello singolarmente. Krajono fornisce la possibilità di attivare
la funzionalità di export specificando l'argomento \texttt{-extract\_dedukti} a Matitac,
sia lavorando con un unico file sia con un'intera directory. A processo concluso 
si potranno trovare, nella directory dei sorgenti, i file \textit{.dk} contenente
il codice matita esportato.


%PRESENTE----------------------------------------------------------------------------------------------------------
Quando il suddetto flag è attivo il motore di Matita avvia l'esportazione chiamando
una funzione di uno dei moduli OCaml importati da Krajono. Questi moduli compiono
delle analisi del tipo e della struttura del termine Matita con lo scopo di
capire come applicare l'encoding appropriato. Una volta costruiti, i termini
Dedukti risultanti dalla codifica vengono inseriti in una tabella hash
che alla fine del processo verrà scritta sul file \textit{.dk} di output.

È importante notare come il processo di esporazione non è sempre un encoding
\textit{uno a uno}, ma determinati tipologie di termini Matita vengono codificati
in più termini Dedukti.

\subsection{L'encoding}
%TODO IDk what to write

\section{Problemi di Krajono} \label{ProblemiKrajono}



\chapter{Da Dedukti a Matita}
Come visto nel capitolo precedente, usando Krajono è possibile esportare 
del codice Matita verso Dedukti, tuttavia non è possibile fare il contrario,
in quanto ne Krajono, ne Dedukti stesso godono di questa funzionalità.
L'export è dunque a senso unico, e qualcosa di esportato non può essere
re-importato in Matita. Il lavoro di questa tesi è proprio il seguente:
rendere Matita capace di esportare ed importare codice da e verso Dedukti.
Con un export a doppio senso gli sviluppatori Matita saranno in grado 
di usare dimostrazioni Dedukti e vice versa.

\section{Il parser Dedukti} \label{parserDedukti}
% TODO
% spiegare che dedukti è diventato dipendenza

\section{Tradurre i termini Dedukti}
Alcuni termini Dedukti sono direttamente traducibili in termini Matita. Alcuni
invece richiedono una logica più complessa.

\subsection{La gestione dei nomi}

\subsection{il resto bla bla}

\paragraph{Costanti}
Per tradurre le costanti è stato necessario semplicemente convertire il nome
Dedukti in un \textit{uri} Matita. Per evitare conflitti con i nomi si è aggiunto
anche una tabella hash che per tenere traccia dei nomi già assegnati e dei 
relativi uri.

\paragraph{Indici di De Brujin}
Entrambi i software fanno uso degli indici di De Brujin, una rappresentazione
compatta delle variabili legate all'interno di un termine. Sono utilizzati per
semplificare la manipolazione dei termini, eliminando la necessità di utilizzare
nomi unici per le variabili e consentendo di eseguire operazioni come la 
sostituzione e il confronto tra termini in modo efficente. 
Essendo un indice rappresentato da un intero la conversione è stata diretta.
L'unica accortezza presa è stata l'aggiungere $1$ ad ogni indice in quanto
Matita fa uso di un sistema \textit{$1$ based} mentre Dedukti conta partendo
da $0$.

\paragraph{$\beta$-riduzione}
In Dedukti il passo di $\beta$-riduzione è rappresentato tramite una tripla
contenente: 
\begin{itemize}
  \item Un termine rappresentante la $\lambda$ astrazione da ridurre
  \item Un termine rappresentante il primo argomento da usare per la riduzione
  \item Una lista di termini rappresentante il resto degli argomenti
\end{itemize}
Per tradurlo è stato sufficiente tradurre individualmente, ricorsivamente, ciascuno
di questi termini, e assemblarli costruendo un oggetto che rappresenta la 
$\beta$-riduzione in Matita. 
L'esportazione del capitolo \ref{capitoloExport} utilizza delle particolari
definizioni per rappresentare alcuni termini di Matita %TODO all the cic.stuff

\paragraph{$\lambda$-astrazione e prodotto}
Astrazioni lambda e prodotti condividono la stessa struttura e pertanto sono
rappresentati allo stesso modo sia in Dedukti che in Matita.
Nel primo sono rappresentati come una tripla contenente
\begin{itemize}
  \item Un identificativo Dedukti
  \item Un termine rappresentante il tipo della $\lambda$-astrazione o del prodotto
  \item Un termine rappresentante il corpo
\end{itemize}
Si costruiscono quindi i relativi oggetti Matita convertendo l'identificativo
e traducendo ricorsivamente il tipo e il corpo.

\paragraph{Type e Kind}
Il calcolo lambda-pi usa i concetti di \textit{type} e \textit{kind}: type è
il classico \textit{concetto di tipo} usato per la classificazione di termini,
mentre kind è un tipo speciale rappresentante il tipo di tutti i tipi. Ad 
esempio 5 può avere tipo $nat$, mentre $nat$ TODO avere kind $*$. Nel 
calcolo delle costruzioni (co)induttive alla base di Matita questi concetti
non sono presenti, e pertanto non sono stati tradotti. %TODO add info about sort and univs

\section{Invertire l'esportazione}
Fino adesso è stato visto come importare in Matita del codice Dedukti semplice,
tuttavia, se si volesse importare del codice precedentemente esportato usando
la funzionalità del capitolo \ref{capitoloExport}, ci si accorgerebbe della 
scomparsa di alcuni costrutti Matita, come ad esempio il \textit{match}. Questo
perché, dato che Dedukti non li possiede, durante l'esportazione sono stati
trasformati in termini che ne emulano il comportamento. Per rendere dunque possibile
la costruzione di un codice quanto più vicino all'originale si ha pensato ed 
implementato la strategia qua successivamente discussa.

\subsection{L'uso delle pragma}
Il linguaggio di Dedukti da la possibilità all'utente di scrivere delle 
\textit{direttive} o \textit{pragma}. Queste sono delle particolari righe di
codice interpretate dal compilatore e che pertanto non fanno parte del programma.
Usandole è possibile istruire il compilatore Dedukti, o nel nostro caso il parser
Dedukti integrato dentro Matita \ref{parserDedukti}, affinché agisca in determinati
modi quando le incontra. 
Nel caso specifico di questa tesi, sono state definite ed usate delle pragma
per indicare quali parti di codice Dedukti che fanno riferimento ad un costrutto 
Matita andato perso durante l'esportazione.

\paragraph{Sintassi} Le pragma sono sostanzialmente delle stringhe, quindi
è stato necessario pensare ad uno standard che aiutasse a strutturarle e 
ne facilitasse il parsing. Dato che la traduzione di un costrutto Matita in 
Dedukti può risultare in un blocco di istruzioni bisogna essere in grado di
capire dove questo inizia e finisce. Inoltre, per poter ricostruire un oggetto
talvolta è necessario salvare degli attributi aggiuntivi, specificando anche
a quale oggetto fanno riferimento. Per tanto si è pensato di inserire pragma
con la seguente sintassi:

\begin{center}
  \texttt{\#PRAGMA [BEGIN|END] <NOME> [ATTR[:rif]=val]... .}
\end{center}
Le seguenti pragma ad esempio sono valide
\begin{itemize}
  \item  \texttt{\#PRAGMA FOO BAR=42.} L'istruzione successiva è di tipo
\texttt{FOO} e ha l'attributo \texttt{BAR} che vale \texttt{42}.
  \item  \texttt{\#PRAGMA BEGIN BLOCK GREETINGS:world=hello.} Inizia un blocco di
    tipo \texttt{BLOCK} con l'attributo \texttt{GREETINGS} di valore \texttt{hello}
    che fa riferimento a \texttt{world}.
  \item  \texttt{\#PRAGMA END BLOCK.} Fine di un blocco di tipo \texttt{BLOCK}
\end{itemize}

\subsection{Punto fisso}
Dato un insieme $A$ e una funzione $f: A \rightarrow A$, $x \in A$ si dice
\textit{punto fisso di f} se e solo se $x = f(x)$. Nella teoria dei tipi 
questo concetto è utile per rappresentare le funzioni ricorsive.
Questo è uno di quei costrutti che viene perso durante l'esportazione, in 
quanto Matita ne fa uso mentre Dedukti no.

\paragraph{Punto fisso nell'encoding} Un'istruzione di tipo punto fisso è tradotta
come un insieme di astrazioni lambda e delle regole di riscrittura. In particolare
per ciascun punto fisso si ottengono un'astrazione e una regola per rappresentare
il tipo e un'astrazione e una regola per rappresentare il corpo.

\paragraph{La pragma} Per poter ricostruire l'oggetto Matita iniziale, oltre
ad individuare tipo e corpo, è necessario anche conoscere un parametro chiamato
\textit{recno}. Questo è un indice (\textit{0-based}) che serve ad individuare
su quale argomento della funzione avviene la ricorsione. Il valore però viene
anch'esso perso durante l'esportazione, quindi è stato necessario trovare un
modo per preservarlo. Per conservare queste informazioni si è modificato il
codice dell'esportazione in modo da inserire delle pragma come delimitatori
del blocco di istruzioni Dedukti rappresentanti l'encoding del fixpoint.
In particolare la pragma per indicare l'inizio del blocco è nella forma:
\begin{center}
\texttt{\#PRAGMA BEGIN FIXPOINT NAME=name RECNO:name=0}
\end{center}
Mentre la pragma di chiusura è nella forma:
\begin{center}
\texttt{\#PRAGMA END FIXPOINT}
\end{center}

L'attributo \texttt{NAME} rappresenta il nome della funzione, e ce ne possono
essere più di uno nel caso in cui si abbia della ricorsione mutua. Il \textit{recno}
viene esplicitato nell'omonimo attributo, specificando anche a quale nome fa
riferimento. Nel caso di ricorsione mutua si potrebbe ottenere una pragma nella
forma:
\begin{center}
\texttt{\#PRAGMA BEGIN FIXPOINT NAME=f NAME=g RECNO:f=0 RECNO:g=1}
\end{center}

\paragraph{Il corpo}

\subsection{Tipi induttivi}

\subsection{pattern matching}

\subsection{Pragma generated}

\chapter{Conclusioni \& sviluppi futuri}

% bio
% http://www.cs.unibo.it/~ricciott/PAPERS/system_description2011_draft.pdf
\end{document}
