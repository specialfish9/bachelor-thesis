\documentclass[12pt,a4paper]{report}
\usepackage[italian]{babel}
\usepackage[utf8]{inputenc}
\usepackage{newlfont}
\usepackage{amsmath}
\usepackage{graphicx}
\usepackage{listings}
\graphicspath{ {./pic/} }

\begin{document}

\begin{titlepage}
  \begin{center}
    {{
      \Large{\textsc{Alma Mater Studiorum $\cdot$ Universit\`a di Bologna}}
    }} \rule[0.1cm]{15.8cm}{0.1mm}
    \rule[0.5cm]{15.8cm}{0.6mm}
    {\small{\bf SCUOLA DI SCIENZE\\
    Corso di Laurea in Informatica}}
  \end{center}
  \vspace{15mm}
  \begin{center}
    {\LARGE{\bf LA MIA FANTASTICA}}\\
    \vspace{3mm}
    {\LARGE{\bf OTTIMISTICA}}\\
    \vspace{3mm}
    {\LARGE{\bf TESI}}\\
  \end{center}
  \vspace{40mm}
  \par
  \noindent
  \begin{minipage}[t]{0.47\textwidth}
  {\large{\bf Relatore:\\
  Chiar.mo Prof.\\ % TODO 
  Claudio Sacerdoti Coen}}
  \end{minipage}
  \hfill
  \begin{minipage}[t]{0.47\textwidth}\raggedleft
  {\large{\bf Presentata da:\\
  Mattia Girolimetto}}
  \end{minipage}
  \vspace{20mm}
  \begin{center}
  {\large{\bf I Appello di Laurea\\%inserire il numero della sessione in cui ci si laurea
  Anno Accademico 2022-2023}}%inserire l'anno accademico a cui si è iscritti
  \end{center}
\end{titlepage}


\tableofcontents

\chapter{Introduzione}

\section{Teoria dei Tipi}

\section{Dimostratori Interattivi di Teoremi}
Un dimostratore interattivo di teoremi (o \textit{proof assistant}) è un software 
che permette all'utente di costruire e verificare delle dimostrazioni matematiche
formali. Presa in input una prova espressa utilizzando uno specifico linguaggio 
formale, simile ad un linguaggio di programmazione, il software è in grado di
verificarne la correttezza. In questo modo si possono costruire dimostrazioni
in modo interattivo, controllando progressivamente la corretezza di ogni passo.
Uno dei benefici chiave dell'usare un dimostratore interattivo automatico è l'
abilità di eliminare gli errori e le ambiguità che possono comparire nelle 
dimostrazioni tradizionali.

\subsection{Matita}
Matita è un  proof assistant sotto sviluppo nel dipartimento di informatica all'
Università di Bologna. E' basato sul \textit{calcolo delle costruzioni coinduttive}.
Il software, che è open source, è scritto nel linguaggio di programmazione OCAML 
ed è rilasciato secondo i termini della GNU General Public Licence.

\subsection{Dedukti}
Dedukti (che significa "dedurre" in esperanto) è un \textit{logical framework}
sviluppato da alcuni ricercatori del INRIA, basato sul \textit{calcolo lambda\-pi}.
Il software è open source, anch'esso scritto nel linguaggio di programmazione OCAML
e distribuito secondo i termini della CeCILL\-B License.
\chapter{Parte tecnica} % TODO

\chapter{Conclusioni}

\chapter{Sviluppi futuri}





\end{document}
