\documentclass[12pta4paper]{report}
\usepackage[italian]{babel}
\usepackage[utf8]{inputenc}
\usepackage{newlfont}
\usepackage{amsmath}
\usepackage{amsfonts}
\usepackage{graphicx}
\usepackage{listings}
\graphicspath{ {./pic/} }

\begin{document}

\chapter*{Introduzione}
I \textit{proof assistant } sono software in grado di verificare una dimostrazione
formale scritta da un utente usando un linguaggio funzionale con un sistema di
tipi avanzato. Il loro sviluppo negli anni ha portato ad una frammentazione
in quanto sistemi diversi che usano logiche diverse non sono compatibili tra 
loro. Per ovviare a questo problema nasce \textit{Dedukti} la cui idea è quella
di agire da \textit{type checker} per definizioni enunciati e prove scritti in
una sua logica codificata in un metalinguaggio standard sufficentemente espressivo 
da poter rappresentare un qualunque altro termine di una qualunque altra logica. Grazie
a questo per gli altri proof assistant come \textit{Coq} e \textit{HOL lite} è possibile sviluppare
un sistema di codifica che converta dimostrazioni scritte nel loro linguaggio 
nel metalinguaggio di Dedukti. Proprio per questo nasce \textit{Krajono} un
fork di \textit{Matita} un proof system  in sviluppo nel Dipartimento 
di Informatica dell'Università di Bologna. La caratteristica che lo contraddistingue
è la possibilità di compiere questo tipo di esportazione. Krajono tuttavia non
è più mantenuto e per tanto non integra le funzionalità del Matita baseline sviluppate negli ultimi anni.
Il lavoro di questa tesi è diviso in due parti: la prima ha previsto l'integrazione
della funzionalità di esportazione di Krajono nel Matita odierno la seconda è
consistita nel implementare la possibilità di re-importare codice Dedukti
esportato precedentemente da Matita. Il secondo punto in particolare ha richiesto
la risoluzione di una serie di problemi dovuti in primis dal fatto che il
linguaggio di Matita è più rigido e strutturato del metalinguaggio di Dedukti
per tanto durante l'esportazione venivano perse informazioni necessarie alla
ricostruzione del codice originale. Ciò è stato risolto dotando l'export di
un meccanismo per preservarle dove possibile. In secondo luogo il metalinguaggio 
di Dedukti non gode di \textit{confluenza} e \textit{normalizzazione} proprietà 
che il linguaggio di Matita garantisce attraverso alcuni dei suoi costruitti.
È stato fondamentale quindi riuscire a ricostruirli per ripristinare queste proprietà.

\end{document}
